\documentclass[]{article}
\usepackage{lmodern}
\usepackage{amssymb,amsmath}
\usepackage{ifxetex,ifluatex}
\usepackage{fixltx2e} % provides \textsubscript
\ifnum 0\ifxetex 1\fi\ifluatex 1\fi=0 % if pdftex
  \usepackage[T1]{fontenc}
  \usepackage[utf8]{inputenc}
\else % if luatex or xelatex
  \ifxetex
    \usepackage{mathspec}
  \else
    \usepackage{fontspec}
  \fi
  \defaultfontfeatures{Ligatures=TeX,Scale=MatchLowercase}
\fi
% use upquote if available, for straight quotes in verbatim environments
\IfFileExists{upquote.sty}{\usepackage{upquote}}{}
% use microtype if available
\IfFileExists{microtype.sty}{%
\usepackage{microtype}
\UseMicrotypeSet[protrusion]{basicmath} % disable protrusion for tt fonts
}{}
\usepackage[margin=1in]{geometry}
\usepackage{hyperref}
\hypersetup{unicode=true,
            pdftitle={LC1 Fitting functions},
            pdfauthor={Harm Manders},
            pdfborder={0 0 0},
            breaklinks=true}
\urlstyle{same}  % don't use monospace font for urls
\usepackage{color}
\usepackage{fancyvrb}
\newcommand{\VerbBar}{|}
\newcommand{\VERB}{\Verb[commandchars=\\\{\}]}
\DefineVerbatimEnvironment{Highlighting}{Verbatim}{commandchars=\\\{\}}
% Add ',fontsize=\small' for more characters per line
\usepackage{framed}
\definecolor{shadecolor}{RGB}{248,248,248}
\newenvironment{Shaded}{\begin{snugshade}}{\end{snugshade}}
\newcommand{\KeywordTok}[1]{\textcolor[rgb]{0.13,0.29,0.53}{\textbf{#1}}}
\newcommand{\DataTypeTok}[1]{\textcolor[rgb]{0.13,0.29,0.53}{#1}}
\newcommand{\DecValTok}[1]{\textcolor[rgb]{0.00,0.00,0.81}{#1}}
\newcommand{\BaseNTok}[1]{\textcolor[rgb]{0.00,0.00,0.81}{#1}}
\newcommand{\FloatTok}[1]{\textcolor[rgb]{0.00,0.00,0.81}{#1}}
\newcommand{\ConstantTok}[1]{\textcolor[rgb]{0.00,0.00,0.00}{#1}}
\newcommand{\CharTok}[1]{\textcolor[rgb]{0.31,0.60,0.02}{#1}}
\newcommand{\SpecialCharTok}[1]{\textcolor[rgb]{0.00,0.00,0.00}{#1}}
\newcommand{\StringTok}[1]{\textcolor[rgb]{0.31,0.60,0.02}{#1}}
\newcommand{\VerbatimStringTok}[1]{\textcolor[rgb]{0.31,0.60,0.02}{#1}}
\newcommand{\SpecialStringTok}[1]{\textcolor[rgb]{0.31,0.60,0.02}{#1}}
\newcommand{\ImportTok}[1]{#1}
\newcommand{\CommentTok}[1]{\textcolor[rgb]{0.56,0.35,0.01}{\textit{#1}}}
\newcommand{\DocumentationTok}[1]{\textcolor[rgb]{0.56,0.35,0.01}{\textbf{\textit{#1}}}}
\newcommand{\AnnotationTok}[1]{\textcolor[rgb]{0.56,0.35,0.01}{\textbf{\textit{#1}}}}
\newcommand{\CommentVarTok}[1]{\textcolor[rgb]{0.56,0.35,0.01}{\textbf{\textit{#1}}}}
\newcommand{\OtherTok}[1]{\textcolor[rgb]{0.56,0.35,0.01}{#1}}
\newcommand{\FunctionTok}[1]{\textcolor[rgb]{0.00,0.00,0.00}{#1}}
\newcommand{\VariableTok}[1]{\textcolor[rgb]{0.00,0.00,0.00}{#1}}
\newcommand{\ControlFlowTok}[1]{\textcolor[rgb]{0.13,0.29,0.53}{\textbf{#1}}}
\newcommand{\OperatorTok}[1]{\textcolor[rgb]{0.81,0.36,0.00}{\textbf{#1}}}
\newcommand{\BuiltInTok}[1]{#1}
\newcommand{\ExtensionTok}[1]{#1}
\newcommand{\PreprocessorTok}[1]{\textcolor[rgb]{0.56,0.35,0.01}{\textit{#1}}}
\newcommand{\AttributeTok}[1]{\textcolor[rgb]{0.77,0.63,0.00}{#1}}
\newcommand{\RegionMarkerTok}[1]{#1}
\newcommand{\InformationTok}[1]{\textcolor[rgb]{0.56,0.35,0.01}{\textbf{\textit{#1}}}}
\newcommand{\WarningTok}[1]{\textcolor[rgb]{0.56,0.35,0.01}{\textbf{\textit{#1}}}}
\newcommand{\AlertTok}[1]{\textcolor[rgb]{0.94,0.16,0.16}{#1}}
\newcommand{\ErrorTok}[1]{\textcolor[rgb]{0.64,0.00,0.00}{\textbf{#1}}}
\newcommand{\NormalTok}[1]{#1}
\usepackage{graphicx,grffile}
\makeatletter
\def\maxwidth{\ifdim\Gin@nat@width>\linewidth\linewidth\else\Gin@nat@width\fi}
\def\maxheight{\ifdim\Gin@nat@height>\textheight\textheight\else\Gin@nat@height\fi}
\makeatother
% Scale images if necessary, so that they will not overflow the page
% margins by default, and it is still possible to overwrite the defaults
% using explicit options in \includegraphics[width, height, ...]{}
\setkeys{Gin}{width=\maxwidth,height=\maxheight,keepaspectratio}
\IfFileExists{parskip.sty}{%
\usepackage{parskip}
}{% else
\setlength{\parindent}{0pt}
\setlength{\parskip}{6pt plus 2pt minus 1pt}
}
\setlength{\emergencystretch}{3em}  % prevent overfull lines
\providecommand{\tightlist}{%
  \setlength{\itemsep}{0pt}\setlength{\parskip}{0pt}}
\setcounter{secnumdepth}{0}
% Redefines (sub)paragraphs to behave more like sections
\ifx\paragraph\undefined\else
\let\oldparagraph\paragraph
\renewcommand{\paragraph}[1]{\oldparagraph{#1}\mbox{}}
\fi
\ifx\subparagraph\undefined\else
\let\oldsubparagraph\subparagraph
\renewcommand{\subparagraph}[1]{\oldsubparagraph{#1}\mbox{}}
\fi

%%% Use protect on footnotes to avoid problems with footnotes in titles
\let\rmarkdownfootnote\footnote%
\def\footnote{\protect\rmarkdownfootnote}

%%% Change title format to be more compact
\usepackage{titling}

% Create subtitle command for use in maketitle
\newcommand{\subtitle}[1]{
  \posttitle{
    \begin{center}\large#1\end{center}
    }
}

\setlength{\droptitle}{-2em}
  \title{LC1 Fitting functions}
  \pretitle{\vspace{\droptitle}\centering\huge}
  \posttitle{\par}
  \author{Harm Manders}
  \preauthor{\centering\large\emph}
  \postauthor{\par}
  \date{}
  \predate{}\postdate{}


\begin{document}
\maketitle

\begin{Shaded}
\begin{Highlighting}[]
\KeywordTok{rm}\NormalTok{(}\DataTypeTok{list=}\KeywordTok{ls}\NormalTok{()) }\CommentTok{# clear the workspace.. always a good idea before you start a new project}
\KeywordTok{set.seed}\NormalTok{(}\DecValTok{321}\NormalTok{) }\CommentTok{# so we will all be talking about the same random results}
\KeywordTok{setwd}\NormalTok{(}\StringTok{"/home/harm/Uni/cogmod/LC1"}\NormalTok{) }\CommentTok{# set to your working dir with the files you need. }
\end{Highlighting}
\end{Shaded}

\section{1 creating a function}\label{creating-a-function}

Creating functions in R is easy. The basic structure is:\\
\emph{function\_name} \textless{}- \textbf{function}(list of variables)
\{

do something with variables

\textbf{return}(another variable)\\
\}

for example:

\begin{Shaded}
\begin{Highlighting}[]
\NormalTok{himmelblau <-}\StringTok{ }\ControlFlowTok{function}\NormalTok{(x,y)\{}
\NormalTok{  h=(x}\OperatorTok{^}\DecValTok{2}\OperatorTok{+}\NormalTok{y}\OperatorTok{-}\DecValTok{11}\NormalTok{)}\OperatorTok{^}\DecValTok{2}\OperatorTok{+}\NormalTok{(x}\OperatorTok{+}\NormalTok{y}\OperatorTok{^}\DecValTok{2}\OperatorTok{-}\DecValTok{7}\NormalTok{)}\OperatorTok{^}\DecValTok{2}
  \KeywordTok{return}\NormalTok{(h)}
\NormalTok{\}}

\CommentTok{# call the function:}
\KeywordTok{himmelblau}\NormalTok{(}\DecValTok{2}\NormalTok{,}\DecValTok{3}\NormalTok{)}
\end{Highlighting}
\end{Shaded}

\begin{verbatim}
## [1] 32
\end{verbatim}

\begin{Shaded}
\begin{Highlighting}[]
\NormalTok{## or just enter a list of variable v = c(x,y)}
\NormalTok{himmelblau <-}\StringTok{ }\ControlFlowTok{function}\NormalTok{(v)\{}
\NormalTok{  x<-v[}\DecValTok{1}\NormalTok{]}
\NormalTok{  y<-v[}\DecValTok{2}\NormalTok{]}
\NormalTok{  h=(x}\OperatorTok{^}\DecValTok{2}\OperatorTok{+}\NormalTok{y}\OperatorTok{-}\DecValTok{11}\NormalTok{)}\OperatorTok{^}\DecValTok{2}\OperatorTok{+}\NormalTok{(x}\OperatorTok{+}\NormalTok{y}\OperatorTok{^}\DecValTok{2}\OperatorTok{-}\DecValTok{7}\NormalTok{)}\OperatorTok{^}\DecValTok{2}
  \KeywordTok{return}\NormalTok{(h)}
\NormalTok{\}}
\CommentTok{# now call the function with a list:}
\KeywordTok{himmelblau}\NormalTok{(}\KeywordTok{c}\NormalTok{(}\DecValTok{2}\NormalTok{,}\DecValTok{3}\NormalTok{))}
\end{Highlighting}
\end{Shaded}

\begin{verbatim}
## [1] 32
\end{verbatim}

\begin{center}\rule{0.5\linewidth}{\linethickness}\end{center}

\section{2 Fitting a function to data}\label{fitting-a-function-to-data}

In this exercise we will fit different functions to data and compare
their fits. In the first step we will generate the data, in the second
step we will fit different functions to it. This exercise will also be
used to illustrate the difference between fit and prediction.\\
***

\subsubsection{2.1 Generating Simulation
data}\label{generating-simulation-data}

\subsubsection{\texorpdfstring{Q1 To begin write your first function:
\[y = e^{\frac{(.3 \cdot x)}{2}}\] and call this function \emph{curve}.
To be clear the input should be \emph{x} and it should return \emph{y}.
(1
point)}{Q1 To begin write your first function: y = e\^{}\{\textbackslash{}frac\{(.3 \textbackslash{}cdot x)\}\{2\}\} and call this function curve. To be clear the input should be x and it should return y. (1 point)}}\label{q1-to-begin-write-your-first-function-y-efrac.3-cdot-x2-and-call-this-function-curve.-to-be-clear-the-input-should-be-x-and-it-should-return-y.-1-point}

\begin{Shaded}
\begin{Highlighting}[]
\NormalTok{curve <-}\StringTok{ }\ControlFlowTok{function}\NormalTok{(x) \{}
  
\NormalTok{  y =}\StringTok{ }\KeywordTok{exp}\NormalTok{((.}\DecValTok{3}\OperatorTok{*}\StringTok{ }\NormalTok{x) }\OperatorTok{/}\StringTok{ }\DecValTok{2}\NormalTok{)}
  \KeywordTok{return}\NormalTok{(y)}
\NormalTok{\}}

\KeywordTok{curve}\NormalTok{(}\DecValTok{2}\OperatorTok{/}\FloatTok{0.3}\NormalTok{)}
\end{Highlighting}
\end{Shaded}

\begin{verbatim}
## [1] 2.718282
\end{verbatim}

\begin{Shaded}
\begin{Highlighting}[]
\CommentTok{# returns e}
\end{Highlighting}
\end{Shaded}

\begin{center}\rule{0.5\linewidth}{\linethickness}\end{center}

\subsubsection{Q2 (save your code, you need the constructed code later)
(3
points)}\label{q2-save-your-code-you-need-the-constructed-code-later-3-points}

The next step is to generate some data using this function:

\begin{enumerate}
\def\labelenumi{\arabic{enumi}.}
\item
  create an empty vector c() called \textbf{sim\_data}
\item
  make and save a list called \textbf{points} with numbers between 1 and
  10 in steps of .5 using \textbf{seq()}
\end{enumerate}

3a. Use the numbers in \textbf{points} to generate y values using your
\textbf{curve()} function and store them in \textbf{sim\_data}

3b. also add noise to the \textbf{sim\_data} from a uniform distribution
between -1.5 to 1.5 (hunt use runif())

\begin{center}\rule{0.5\linewidth}{\linethickness}\end{center}

\begin{Shaded}
\begin{Highlighting}[]
\NormalTok{sim_data =}\StringTok{ }\KeywordTok{c}\NormalTok{()}
\NormalTok{points =}\StringTok{ }\KeywordTok{seq}\NormalTok{(}\DecValTok{1}\NormalTok{, }\DecValTok{10}\NormalTok{, }\FloatTok{0.5}\NormalTok{)}
\NormalTok{sim_data =}\StringTok{ }\KeywordTok{curve}\NormalTok{(points)}
\NormalTok{noise =}\StringTok{ }\KeywordTok{runif}\NormalTok{(}\KeywordTok{length}\NormalTok{(points), }\OperatorTok{-}\FloatTok{1.5}\NormalTok{, }\FloatTok{1.5}\NormalTok{)}
\NormalTok{sim_data =}\StringTok{ }\NormalTok{sim_data }\OperatorTok{+}\StringTok{ }\NormalTok{noise}
\KeywordTok{plot}\NormalTok{(points, sim_data)}
\end{Highlighting}
\end{Shaded}

\includegraphics{LC1_2_fitting_functions_1_files/figure-latex/unnamed-chunk-4-1.pdf}

Now we will create a dataframe with these x and y values called
\textbf{dat\_full} and a subset of that data called \textbf{dat\_short}
:

\begin{Shaded}
\begin{Highlighting}[]
\CommentTok{# create a dataframe}
\NormalTok{dat_full<-}\KeywordTok{as.data.frame}\NormalTok{(}\KeywordTok{cbind}\NormalTok{(points,sim_data))}
\KeywordTok{colnames}\NormalTok{(dat_full)<-}\KeywordTok{c}\NormalTok{(}\StringTok{"x"}\NormalTok{,}\StringTok{"y"}\NormalTok{)}
\CommentTok{# create a subset for illstruation}
\NormalTok{dat_short<-}\KeywordTok{subset}\NormalTok{(dat_full,dat_full}\OperatorTok{$}\NormalTok{x}\OperatorTok{<}\DecValTok{6}\NormalTok{)}
\end{Highlighting}
\end{Shaded}

\begin{center}\rule{0.5\linewidth}{\linethickness}\end{center}

\section{Generating curves to fit to the
data:}\label{generating-curves-to-fit-to-the-data}

Next, we will generate a few curves that will then fit to the data we
just generated. The output of the functions that we will write will be
the \textbf{Mean Squared Error} or \textbf{MSE} which is what is says.

The function will:

1.take in the values for \textbf{x} and \textbf{y}

\begin{enumerate}
\def\labelenumi{\arabic{enumi}.}
\setcounter{enumi}{1}
\item
  generate estimations or prediction of \textbf{y}, called
  \textbf{y\_hat} using a specific function (in this example linear)
\item
  compare the estimated \textbf{y\_hat} with the true \textbf{y}, the
  difference is called the \textbf{error}
\item
  MSE is the mean of the squared error
\end{enumerate}

\begin{Shaded}
\begin{Highlighting}[]
\NormalTok{## creating a simple linear function:}
\NormalTok{linear <-}\StringTok{ }\ControlFlowTok{function}\NormalTok{(data,v)\{}
\NormalTok{  b1<-v}
\NormalTok{  y_hat=b1}\OperatorTok{*}\NormalTok{data}\OperatorTok{$}\NormalTok{x}
\NormalTok{  error =}\StringTok{ }\NormalTok{y_hat}\OperatorTok{-}\NormalTok{data}\OperatorTok{$}\NormalTok{y}
\NormalTok{  MSE =}\StringTok{ }\KeywordTok{mean}\NormalTok{(error}\OperatorTok{^}\DecValTok{2}\NormalTok{)}
  \KeywordTok{return}\NormalTok{(MSE)}
\NormalTok{\}}
\end{Highlighting}
\end{Shaded}

\begin{center}\rule{0.5\linewidth}{\linethickness}\end{center}

\subsubsection{Q3 Generate a polynomial function to fit to the data: (1
point)}\label{q3-generate-a-polynomial-function-to-fit-to-the-data-1-point}

\begin{center}\rule{0.5\linewidth}{\linethickness}\end{center}

Generate a polynomial function: \[y=b0+b1*x+b2*x^2+b3*x^3+b4*x^4\]

\begin{Shaded}
\begin{Highlighting}[]
\NormalTok{polynomial <-}\StringTok{ }\ControlFlowTok{function}\NormalTok{(data, v) \{}
\NormalTok{  x =}\StringTok{ }\NormalTok{data}\OperatorTok{$}\NormalTok{x}
\NormalTok{  y_hat =}\StringTok{ }\NormalTok{v[}\DecValTok{1}\NormalTok{] }\OperatorTok{+}\StringTok{ }\NormalTok{v[}\DecValTok{2}\NormalTok{]}\OperatorTok{*}\NormalTok{x }\OperatorTok{+}\StringTok{ }\NormalTok{v[}\DecValTok{3}\NormalTok{]}\OperatorTok{*}\NormalTok{x}\OperatorTok{^}\DecValTok{2} \OperatorTok{+}\StringTok{ }\NormalTok{v[}\DecValTok{4}\NormalTok{]}\OperatorTok{*}\NormalTok{x}\OperatorTok{^}\DecValTok{3} \OperatorTok{+}\StringTok{ }\NormalTok{v[}\DecValTok{5}\NormalTok{]}\OperatorTok{*}\NormalTok{x}\OperatorTok{^}\DecValTok{4}
\NormalTok{  error =}\StringTok{ }\NormalTok{y_hat }\OperatorTok{-}\StringTok{ }\NormalTok{data}\OperatorTok{$}\NormalTok{y}
\NormalTok{  MSE =}\StringTok{ }\KeywordTok{mean}\NormalTok{(error}\OperatorTok{^}\DecValTok{2}\NormalTok{)}
  \KeywordTok{return}\NormalTok{(MSE)}
\NormalTok{\}}
\end{Highlighting}
\end{Shaded}

\section{Fitting functions}\label{fitting-functions}

We can use the standard optimization toolbox \textbf{optim} to find the
best fitting parameters for a function. The optimization tool box
requires us to provide:\\
- starting values for the parameters\\
- function\_name\\
- data to fit function on\\
- optimization method - possibly upper and lower bounds for the
parameters

What the optim toolbox will do is try to find the parameters settings
that correspond the minimum value that is returned by the function that
is being optimized. In our case it is trying to minimize the Mean
Squared Error. It is beyond the scope of this course to discuss the
different methods, but they all minimize the error (find the global
minimum).

Let's run through an example:\\
- starting value = 0\\
- function = linear - data = dat\_short - method = ``Brent'' (only one
you can choose if you have just one parameter) - upper and lower bound
values = 10 and -10

results are stored in results1

\begin{Shaded}
\begin{Highlighting}[]
\NormalTok{## perform some analyses on the short_data}

\NormalTok{results1<-}\KeywordTok{optim}\NormalTok{(}\KeywordTok{c}\NormalTok{(}\DecValTok{0}\NormalTok{),linear,}\DataTypeTok{data=}\NormalTok{dat_short,}\DataTypeTok{method=}\StringTok{"Brent"}\NormalTok{, }\DataTypeTok{upper=}\DecValTok{10}\NormalTok{, }\DataTypeTok{lower=}\OperatorTok{-}\DecValTok{10}\NormalTok{)}
\NormalTok{results1}
\end{Highlighting}
\end{Shaded}

\begin{verbatim}
## $par
## [1] 0.4550293
## 
## $value
## [1] 0.9408333
## 
## $counts
## function gradient 
##       NA       NA 
## 
## $convergence
## [1] 0
## 
## $message
## NULL
\end{verbatim}

When you look at the results1 you get the parameter estimate:

\begin{Shaded}
\begin{Highlighting}[]
\NormalTok{results1}\OperatorTok{$}\NormalTok{par}
\end{Highlighting}
\end{Shaded}

\begin{verbatim}
## [1] 0.4550293
\end{verbatim}

But you also get the corresponding MSE:

\begin{Shaded}
\begin{Highlighting}[]
\NormalTok{results1}\OperatorTok{$}\NormalTok{value}
\end{Highlighting}
\end{Shaded}

\begin{verbatim}
## [1] 0.9408333
\end{verbatim}

\subsection{Visualizing the Fit}\label{visualizing-the-fit}

to create residual (error) lines in plot we will create an empty list
and fill it with the predicted y\_hat values for each x value in the
data set, given the parameter we just estimated results1\$par:

\begin{Shaded}
\begin{Highlighting}[]
\NormalTok{fit_data<-}\KeywordTok{c}\NormalTok{()}
\ControlFlowTok{for}\NormalTok{(i }\ControlFlowTok{in} \DecValTok{1}\OperatorTok{:}\KeywordTok{nrow}\NormalTok{(dat_short))\{}
\NormalTok{  fit_data[i]<-results1}\OperatorTok{$}\NormalTok{par[}\DecValTok{1}\NormalTok{]}\OperatorTok{*}\NormalTok{dat_short}\OperatorTok{$}\NormalTok{x[i]}
\NormalTok{\}}
\NormalTok{dat_short}\OperatorTok{$}\NormalTok{error1<-fit_data}
\end{Highlighting}
\end{Shaded}

Next we plot the data

\begin{Shaded}
\begin{Highlighting}[]
\CommentTok{# simply use the plot() function to plot the true data}
\KeywordTok{plot}\NormalTok{(dat_short}\OperatorTok{$}\NormalTok{x,dat_short}\OperatorTok{$}\NormalTok{y)}
\CommentTok{# add a line with lines() that corresponds with the predicted y_hat. }
\CommentTok{# to do so you need to invoke the linear function b1*x and use the estimated b1}
\CommentTok{# so in this case: results1$par[1]*dat_short$x}
\KeywordTok{lines}\NormalTok{(dat_short}\OperatorTok{$}\NormalTok{x,results1}\OperatorTok{$}\NormalTok{par[}\DecValTok{1}\NormalTok{]}\OperatorTok{*}\NormalTok{dat_short}\OperatorTok{$}\NormalTok{x,}\DataTypeTok{col=}\StringTok{"green"}\NormalTok{,}\DataTypeTok{lwd=}\DecValTok{2}\NormalTok{)}
\CommentTok{# here we add segments that indicate the distance or error from the model to the data:  }
\KeywordTok{segments}\NormalTok{(dat_short}\OperatorTok{$}\NormalTok{x,dat_short}\OperatorTok{$}\NormalTok{y,dat_short}\OperatorTok{$}\NormalTok{x,dat_short}\OperatorTok{$}\NormalTok{error1, }\DataTypeTok{col=}\StringTok{"red"}\NormalTok{,}\DataTypeTok{lwd=}\DecValTok{2}\NormalTok{)}
\end{Highlighting}
\end{Shaded}

\includegraphics{LC1_2_fitting_functions_1_files/figure-latex/unnamed-chunk-12-1.pdf}

\subsubsection{Q4 Now fit the polynomial function to the data and
compare the model
fits}\label{q4-now-fit-the-polynomial-function-to-the-data-and-compare-the-model-fits}

\begin{itemize}
\tightlist
\item
  Fit the model as above, and display the best fitted lines in a plots
  (2 points)
\item
  Compare the MSE of the best fit for each model and indicate which is
  the best fitting one. Why do you think did the best fitting model won?
\end{itemize}

\textbf{Hint 1}: use the method=``Nelder-Mead'', it does not require
definitions of upper and lower bounds\\
\textbf{Hint 2}: for the starting values use c c(-5,9,-4,1,.01)

\begin{Shaded}
\begin{Highlighting}[]
\NormalTok{results2<-}\KeywordTok{optim}\NormalTok{(}\KeywordTok{c}\NormalTok{(}\OperatorTok{-}\DecValTok{5}\NormalTok{,}\DecValTok{9}\NormalTok{,}\OperatorTok{-}\DecValTok{4}\NormalTok{,}\DecValTok{1}\NormalTok{,.}\DecValTok{01}\NormalTok{),polynomial,}\DataTypeTok{data=}\NormalTok{dat_short,}\DataTypeTok{method=}\StringTok{"Nelder-Mead"}\NormalTok{)}
\NormalTok{fit_data<-}\KeywordTok{c}\NormalTok{()}
\NormalTok{v =}\StringTok{ }\NormalTok{results2}\OperatorTok{$}\NormalTok{par}
\ControlFlowTok{for}\NormalTok{(i }\ControlFlowTok{in} \DecValTok{1}\OperatorTok{:}\KeywordTok{nrow}\NormalTok{(dat_short))\{}
\NormalTok{  fit_data[i]<-v[}\DecValTok{1}\NormalTok{] }\OperatorTok{+}\StringTok{ }\NormalTok{v[}\DecValTok{2}\NormalTok{]}\OperatorTok{*}\NormalTok{dat_short}\OperatorTok{$}\NormalTok{x[i] }\OperatorTok{+}\StringTok{ }\NormalTok{v[}\DecValTok{3}\NormalTok{]}\OperatorTok{*}\NormalTok{dat_short}\OperatorTok{$}\NormalTok{x[i]}\OperatorTok{^}\DecValTok{2} \OperatorTok{+}\StringTok{ }\NormalTok{v[}\DecValTok{4}\NormalTok{]}\OperatorTok{*}\NormalTok{dat_short}\OperatorTok{$}\NormalTok{x[i]}\OperatorTok{^}\DecValTok{3} \OperatorTok{+}\StringTok{ }\NormalTok{v[}\DecValTok{5}\NormalTok{]}\OperatorTok{*}\NormalTok{dat_short}\OperatorTok{$}\NormalTok{x[i]}\OperatorTok{^}\DecValTok{4}
\NormalTok{\}}
\NormalTok{dat_short}\OperatorTok{$}\NormalTok{error2<-fit_data}
\KeywordTok{plot}\NormalTok{(dat_short}\OperatorTok{$}\NormalTok{x,dat_short}\OperatorTok{$}\NormalTok{y)}
\NormalTok{x =}\StringTok{ }\NormalTok{dat_short}\OperatorTok{$}\NormalTok{x}
\NormalTok{y_hat =}\StringTok{ }\NormalTok{v[}\DecValTok{1}\NormalTok{] }\OperatorTok{+}\StringTok{ }\NormalTok{v[}\DecValTok{2}\NormalTok{]}\OperatorTok{*}\NormalTok{x }\OperatorTok{+}\StringTok{ }\NormalTok{v[}\DecValTok{3}\NormalTok{]}\OperatorTok{*}\NormalTok{x}\OperatorTok{^}\DecValTok{2} \OperatorTok{+}\StringTok{ }\NormalTok{v[}\DecValTok{4}\NormalTok{]}\OperatorTok{*}\NormalTok{x}\OperatorTok{^}\DecValTok{3} \OperatorTok{+}\StringTok{ }\NormalTok{v[}\DecValTok{5}\NormalTok{]}\OperatorTok{*}\NormalTok{x}\OperatorTok{^}\DecValTok{4}

\KeywordTok{lines}\NormalTok{(dat_short}\OperatorTok{$}\NormalTok{x,y_hat,}\DataTypeTok{col=}\StringTok{"green"}\NormalTok{,}\DataTypeTok{lwd=}\DecValTok{2}\NormalTok{)}
\KeywordTok{segments}\NormalTok{(dat_short}\OperatorTok{$}\NormalTok{x,dat_short}\OperatorTok{$}\NormalTok{y,dat_short}\OperatorTok{$}\NormalTok{x,dat_short}\OperatorTok{$}\NormalTok{error2, }\DataTypeTok{col=}\StringTok{"red"}\NormalTok{,}\DataTypeTok{lwd=}\DecValTok{2}\NormalTok{)}
\end{Highlighting}
\end{Shaded}

\includegraphics{LC1_2_fitting_functions_1_files/figure-latex/unnamed-chunk-13-1.pdf}

\section{Cross-Validation}\label{cross-validation}

We have fit the models only using part of the data (dat\_short), now we
can see how the models perform if we use them to predict the data points
that we have left out. This is the very basic principle of
cross-validation. We first fit the model on part of the data and then
see how it performs on a separate set of data (validation). In practice
we may use slightly different strategies to do this, but the principle
is the same.

\subsubsection{Q5 check the goodness of fit for the full data set using
the parameter estimates derived from the small datasets. Describe which
model fits the data the best and explain why, also plot the results to
illustrate your point. (2
points)}\label{q5-check-the-goodness-of-fit-for-the-full-data-set-using-the-parameter-estimates-derived-from-the-small-datasets.-describe-which-model-fits-the-data-the-best-and-explain-why-also-plot-the-results-to-illustrate-your-point.-2-points}

\textbf{hint 1}: just use same script as before but change the data file
to dat\_full. \textbf{hint 2}: use ylim = c(-2,12)

\begin{Shaded}
\begin{Highlighting}[]
\NormalTok{results2<-}\KeywordTok{optim}\NormalTok{(}\KeywordTok{c}\NormalTok{(}\OperatorTok{-}\DecValTok{5}\NormalTok{,}\DecValTok{9}\NormalTok{,}\OperatorTok{-}\DecValTok{4}\NormalTok{,}\DecValTok{1}\NormalTok{,.}\DecValTok{01}\NormalTok{),polynomial,}\DataTypeTok{data=}\NormalTok{dat_short,}\DataTypeTok{method=}\StringTok{"Nelder-Mead"}\NormalTok{)}
\NormalTok{fit_data<-}\KeywordTok{c}\NormalTok{()}
\NormalTok{v =}\StringTok{ }\NormalTok{results2}\OperatorTok{$}\NormalTok{par}
\ControlFlowTok{for}\NormalTok{(i }\ControlFlowTok{in} \DecValTok{1}\OperatorTok{:}\KeywordTok{nrow}\NormalTok{(dat_full))\{}
\NormalTok{  fit_data[i]<-v[}\DecValTok{1}\NormalTok{] }\OperatorTok{+}\StringTok{ }\NormalTok{v[}\DecValTok{2}\NormalTok{]}\OperatorTok{*}\NormalTok{dat_full}\OperatorTok{$}\NormalTok{x[i] }\OperatorTok{+}\StringTok{ }\NormalTok{v[}\DecValTok{3}\NormalTok{]}\OperatorTok{*}\NormalTok{dat_full}\OperatorTok{$}\NormalTok{x[i]}\OperatorTok{^}\DecValTok{2} \OperatorTok{+}\StringTok{ }\NormalTok{v[}\DecValTok{4}\NormalTok{]}\OperatorTok{*}\NormalTok{dat_full}\OperatorTok{$}\NormalTok{x[i]}\OperatorTok{^}\DecValTok{3} \OperatorTok{+}\StringTok{ }\NormalTok{v[}\DecValTok{5}\NormalTok{]}\OperatorTok{*}\NormalTok{dat_full}\OperatorTok{$}\NormalTok{x[i]}\OperatorTok{^}\DecValTok{4}
\NormalTok{\}}
\NormalTok{dat_full}\OperatorTok{$}\NormalTok{error2<-fit_data}
\KeywordTok{plot}\NormalTok{(dat_full}\OperatorTok{$}\NormalTok{x,dat_full}\OperatorTok{$}\NormalTok{y, }\DataTypeTok{ylim=}\KeywordTok{c}\NormalTok{(}\OperatorTok{-}\DecValTok{5}\NormalTok{,}\DecValTok{10}\NormalTok{))}
\NormalTok{x =}\StringTok{ }\NormalTok{dat_full}\OperatorTok{$}\NormalTok{x}
\NormalTok{y_hat =}\StringTok{ }\NormalTok{v[}\DecValTok{1}\NormalTok{] }\OperatorTok{+}\StringTok{ }\NormalTok{v[}\DecValTok{2}\NormalTok{]}\OperatorTok{*}\NormalTok{x }\OperatorTok{+}\StringTok{ }\NormalTok{v[}\DecValTok{3}\NormalTok{]}\OperatorTok{*}\NormalTok{x}\OperatorTok{^}\DecValTok{2} \OperatorTok{+}\StringTok{ }\NormalTok{v[}\DecValTok{4}\NormalTok{]}\OperatorTok{*}\NormalTok{x}\OperatorTok{^}\DecValTok{3} \OperatorTok{+}\StringTok{ }\NormalTok{v[}\DecValTok{5}\NormalTok{]}\OperatorTok{*}\NormalTok{x}\OperatorTok{^}\DecValTok{4}

\KeywordTok{lines}\NormalTok{(dat_full}\OperatorTok{$}\NormalTok{x,y_hat,}\DataTypeTok{col=}\StringTok{"green"}\NormalTok{,}\DataTypeTok{lwd=}\DecValTok{2}\NormalTok{)}
\KeywordTok{segments}\NormalTok{(dat_full}\OperatorTok{$}\NormalTok{x,dat_full}\OperatorTok{$}\NormalTok{y,dat_full}\OperatorTok{$}\NormalTok{x,dat_full}\OperatorTok{$}\NormalTok{error2, }\DataTypeTok{col=}\StringTok{"red"}\NormalTok{,}\DataTypeTok{lwd=}\DecValTok{2}\NormalTok{)}
\end{Highlighting}
\end{Shaded}

\includegraphics{LC1_2_fitting_functions_1_files/figure-latex/unnamed-chunk-14-1.pdf}

\section{2. Fitting Ebbinghaus data.}\label{fitting-ebbinghaus-data.}

Now let's look at some real data

Ebbinghaus proposed the following equation to best described how savings
change over time:\\
\[  Q(t) = u1 / ((log(t))^a1 + u1), \] where the u1 and a1 are free
parameters and log is taken with base 10. Q(t) are the savings at time
interval t. Note that he calculated the best fitting parameters by hand,
using \textbf{minutes} for the time interval. According to these
calculations he suggested that u1= 1.84 and a1 = 1.25 were the best
fitting parameters.

\subsection{\texorpdfstring{Please load the Ebbinghaus data
(``Ebbinghaus.txt''), and see if you can fit the same model to his
original data and see if you get the same parameters. Note the intervals
in this file are in
\textbf{seconds}.}{Please load the Ebbinghaus data (Ebbinghaus.txt), and see if you can fit the same model to his original data and see if you get the same parameters. Note the intervals in this file are in seconds.}}\label{please-load-the-ebbinghaus-data-ebbinghaus.txt-and-see-if-you-can-fit-the-same-model-to-his-original-data-and-see-if-you-get-the-same-parameters.-note-the-intervals-in-this-file-are-in-seconds.}

Later the simpler exponential function has been proposed which is
described by the equation\\
\[  Q(t)=(1+u1*t)^-a1 \]\\
where Q(t) is savings at time t and μ1 and a1 are parameters.

first lets import some real data:

\begin{Shaded}
\begin{Highlighting}[]
\CommentTok{#importing data}
\NormalTok{ebbing_data <-}\StringTok{ }\KeywordTok{read.delim}\NormalTok{(}\StringTok{"Ebbinghaus.txt"}\NormalTok{,}\DataTypeTok{header=} \OtherTok{TRUE}\NormalTok{, }\DataTypeTok{sep =} \StringTok{"}\CharTok{\textbackslash{}t}\StringTok{"}\NormalTok{) }\CommentTok{# open a file, will be stored as data.frame}
\CommentTok{# ebbing_data$Interval_s}
\end{Highlighting}
\end{Shaded}

\subsubsection{Q6 Check if the exponential indeed fits better, report
the fit and the parameters. Show in a plot(s) how the best fitting
models fit the data points. (3 points, 2 for correct fits, 1 for
plot)}\label{q6-check-if-the-exponential-indeed-fits-better-report-the-fit-and-the-parameters.-show-in-a-plots-how-the-best-fitting-models-fit-the-data-points.-3-points-2-for-correct-fits-1-for-plot}

\textbf{hint} : use method=``L-BFGS-B'', upper=10, lower=-10 and c(2,1)
as starting values

\begin{Shaded}
\begin{Highlighting}[]
\NormalTok{ebbinghaus =}\StringTok{ }\ControlFlowTok{function}\NormalTok{(data, v) \{}
\NormalTok{  x =}\StringTok{ }\NormalTok{data}\OperatorTok{$}\NormalTok{Interval_s}
\NormalTok{  y_hat =}\StringTok{ }\NormalTok{v[}\DecValTok{1}\NormalTok{] }\OperatorTok{/}\StringTok{ }\NormalTok{((}\KeywordTok{log}\NormalTok{(x))}\OperatorTok{^}\NormalTok{v[}\DecValTok{2}\NormalTok{] }\OperatorTok{+}\StringTok{ }\NormalTok{v[}\DecValTok{1}\NormalTok{])}
\NormalTok{  error =}\StringTok{ }\NormalTok{y_hat }\OperatorTok{-}\StringTok{ }\NormalTok{data}\OperatorTok{$}\NormalTok{Ebbinghaus}
\NormalTok{  MSE =}\StringTok{ }\KeywordTok{mean}\NormalTok{(error}\OperatorTok{^}\DecValTok{2}\NormalTok{)}
  \KeywordTok{return}\NormalTok{(MSE)}
\NormalTok{\}}

\NormalTok{ebbing_results<-}\KeywordTok{optim}\NormalTok{(}\KeywordTok{c}\NormalTok{(}\FloatTok{1.84}\NormalTok{, }\FloatTok{1.25}\NormalTok{),ebbinghaus,}\DataTypeTok{data=}\NormalTok{ebbing_data,}\DataTypeTok{method=}\StringTok{"L-BFGS-B"}\NormalTok{, }\DataTypeTok{upper =} \DecValTok{10}\NormalTok{, }\DataTypeTok{lower =} \OperatorTok{-}\DecValTok{10}\NormalTok{)}

\NormalTok{fit_data<-}\KeywordTok{c}\NormalTok{()}
\NormalTok{v =}\StringTok{ }\NormalTok{ebbing_results}\OperatorTok{$}\NormalTok{par}
\ControlFlowTok{for}\NormalTok{(i }\ControlFlowTok{in} \DecValTok{1}\OperatorTok{:}\KeywordTok{nrow}\NormalTok{(ebbing_data))\{}
\NormalTok{  fit_data[i]<-}\StringTok{ }\NormalTok{v[}\DecValTok{1}\NormalTok{] }\OperatorTok{/}\StringTok{ }\NormalTok{((}\KeywordTok{log}\NormalTok{(ebbing_data}\OperatorTok{$}\NormalTok{Interval_s[i]))}\OperatorTok{^}\NormalTok{v[}\DecValTok{2}\NormalTok{] }\OperatorTok{+}\StringTok{ }\NormalTok{v[}\DecValTok{1}\NormalTok{])}
\NormalTok{\}}
\NormalTok{ebbing_data}\OperatorTok{$}\NormalTok{error1<-fit_data}
\NormalTok{x =}\StringTok{ }\NormalTok{ebbing_data}\OperatorTok{$}\NormalTok{Interval_s}
\NormalTok{y =}\StringTok{ }\NormalTok{ebbing_data}\OperatorTok{$}\NormalTok{Ebbinghaus}

\KeywordTok{plot}\NormalTok{(ebbing_data}\OperatorTok{$}\NormalTok{Interval_s, ebbing_data}\OperatorTok{$}\NormalTok{Ebbinghaus)}
\NormalTok{y_hat =}\StringTok{ }\NormalTok{v[}\DecValTok{1}\NormalTok{] }\OperatorTok{/}\StringTok{ }\NormalTok{(}\KeywordTok{log}\NormalTok{(ebbing_data}\OperatorTok{$}\NormalTok{Interval_s)}\OperatorTok{^}\NormalTok{v[}\DecValTok{2}\NormalTok{] }\OperatorTok{+}\StringTok{ }\NormalTok{v[}\DecValTok{1}\NormalTok{])}
\KeywordTok{lines}\NormalTok{(ebbing_data}\OperatorTok{$}\NormalTok{Interval_s, y_hat, }\DataTypeTok{col=}\StringTok{"green"}\NormalTok{, }\DataTypeTok{lwd=}\DecValTok{2}\NormalTok{)}
\KeywordTok{segments}\NormalTok{(ebbing_data}\OperatorTok{$}\NormalTok{Interval_s, ebbing_data}\OperatorTok{$}\NormalTok{Ebbinghaus, ebbing_data}\OperatorTok{$}\NormalTok{Interval_s, ebbing_data}\OperatorTok{$}\NormalTok{error1, }\DataTypeTok{col=}\StringTok{"red"}\NormalTok{,}\DataTypeTok{lwd=}\DecValTok{2}\NormalTok{)}
\end{Highlighting}
\end{Shaded}

\includegraphics{LC1_2_fitting_functions_1_files/figure-latex/unnamed-chunk-16-1.pdf}

\begin{Shaded}
\begin{Highlighting}[]
\NormalTok{ebbinghaus_simple =}\StringTok{ }\ControlFlowTok{function}\NormalTok{(data, v) \{}
\NormalTok{  x =}\StringTok{ }\NormalTok{data}\OperatorTok{$}\NormalTok{Interval_s}
\NormalTok{  y_hat =}\StringTok{ }\NormalTok{(}\DecValTok{1} \OperatorTok{+}\StringTok{ }\NormalTok{v[}\DecValTok{1}\NormalTok{] }\OperatorTok{+}\StringTok{ }\NormalTok{x)}\OperatorTok{^}\NormalTok{(}\OperatorTok{-}\NormalTok{v[}\DecValTok{2}\NormalTok{])}
\NormalTok{  error =}\StringTok{ }\NormalTok{y_hat }\OperatorTok{-}\StringTok{ }\NormalTok{data}\OperatorTok{$}\NormalTok{Ebbinghaus}
\NormalTok{  MSE =}\StringTok{ }\KeywordTok{mean}\NormalTok{(error}\OperatorTok{^}\DecValTok{2}\NormalTok{)}
  \KeywordTok{return}\NormalTok{(MSE)}
\NormalTok{\}}

\NormalTok{ebbing_results<-}\KeywordTok{optim}\NormalTok{(}\KeywordTok{c}\NormalTok{(}\FloatTok{1.84}\NormalTok{, }\FloatTok{1.25}\NormalTok{),ebbinghaus_simple,}\DataTypeTok{data=}\NormalTok{ebbing_data,}\DataTypeTok{method=}\StringTok{"L-BFGS-B"}\NormalTok{, }\DataTypeTok{upper =} \DecValTok{10}\NormalTok{, }\DataTypeTok{lower =} \OperatorTok{-}\DecValTok{10}\NormalTok{)}

\NormalTok{fit_data<-}\KeywordTok{c}\NormalTok{()}
\NormalTok{v =}\StringTok{ }\NormalTok{ebbing_results}\OperatorTok{$}\NormalTok{par}
\ControlFlowTok{for}\NormalTok{(i }\ControlFlowTok{in} \DecValTok{1}\OperatorTok{:}\KeywordTok{nrow}\NormalTok{(ebbing_data))\{}
\NormalTok{  fit_data[i]<-}\StringTok{ }\NormalTok{(}\DecValTok{1} \OperatorTok{+}\StringTok{ }\NormalTok{v[}\DecValTok{1}\NormalTok{] }\OperatorTok{+}\StringTok{ }\NormalTok{ebbing_data}\OperatorTok{$}\NormalTok{Interval_s[i])}\OperatorTok{^}\NormalTok{(}\OperatorTok{-}\NormalTok{v[}\DecValTok{2}\NormalTok{])}
\NormalTok{\}}
\NormalTok{ebbing_data}\OperatorTok{$}\NormalTok{error1<-fit_data}
\NormalTok{x =}\StringTok{ }\NormalTok{ebbing_data}\OperatorTok{$}\NormalTok{Interval_s}
\NormalTok{y =}\StringTok{ }\NormalTok{ebbing_data}\OperatorTok{$}\NormalTok{Ebbinghaus}

\KeywordTok{plot}\NormalTok{(ebbing_data}\OperatorTok{$}\NormalTok{Interval_s, ebbing_data}\OperatorTok{$}\NormalTok{Ebbinghaus)}
\NormalTok{y_hat =}\StringTok{ }\NormalTok{(}\DecValTok{1} \OperatorTok{+}\StringTok{ }\NormalTok{v[}\DecValTok{1}\NormalTok{] }\OperatorTok{+}\StringTok{ }\NormalTok{ebbing_data}\OperatorTok{$}\NormalTok{Interval_s)}\OperatorTok{^}\NormalTok{(}\OperatorTok{-}\NormalTok{v[}\DecValTok{2}\NormalTok{])}
\KeywordTok{lines}\NormalTok{(ebbing_data}\OperatorTok{$}\NormalTok{Interval_s, y_hat, }\DataTypeTok{col=}\StringTok{"green"}\NormalTok{, }\DataTypeTok{lwd=}\DecValTok{2}\NormalTok{)}
\KeywordTok{segments}\NormalTok{(ebbing_data}\OperatorTok{$}\NormalTok{Interval_s, ebbing_data}\OperatorTok{$}\NormalTok{Ebbinghaus, ebbing_data}\OperatorTok{$}\NormalTok{Interval_s, ebbing_data}\OperatorTok{$}\NormalTok{error1, }\DataTypeTok{col=}\StringTok{"red"}\NormalTok{,}\DataTypeTok{lwd=}\DecValTok{2}\NormalTok{)}
\end{Highlighting}
\end{Shaded}

\includegraphics{LC1_2_fitting_functions_1_files/figure-latex/unnamed-chunk-17-1.pdf}


\end{document}
